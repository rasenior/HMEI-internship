% Options for packages loaded elsewhere
\PassOptionsToPackage{unicode}{hyperref}
\PassOptionsToPackage{hyphens}{url}
%
\documentclass[
  12pt,
  a4paper,
]{article}
\title{No exit: the conservation reliance of species with too little habitat}
\usepackage{etoolbox}
\makeatletter
\providecommand{\subtitle}[1]{% add subtitle to \maketitle
  \apptocmd{\@title}{\par {\large #1 \par}}{}{}
}
\makeatother
\subtitle{Supplementary Information}
\author{}
\date{\vspace{-2.5em}}

\usepackage{amsmath,amssymb}
\usepackage{lmodern}
\usepackage{iftex}
\ifPDFTeX
  \usepackage[T1]{fontenc}
  \usepackage[utf8]{inputenc}
  \usepackage{textcomp} % provide euro and other symbols
\else % if luatex or xetex
  \usepackage{unicode-math}
  \defaultfontfeatures{Scale=MatchLowercase}
  \defaultfontfeatures[\rmfamily]{Ligatures=TeX,Scale=1}
\fi
% Use upquote if available, for straight quotes in verbatim environments
\IfFileExists{upquote.sty}{\usepackage{upquote}}{}
\IfFileExists{microtype.sty}{% use microtype if available
  \usepackage[]{microtype}
  \UseMicrotypeSet[protrusion]{basicmath} % disable protrusion for tt fonts
}{}
\makeatletter
\@ifundefined{KOMAClassName}{% if non-KOMA class
  \IfFileExists{parskip.sty}{%
    \usepackage{parskip}
  }{% else
    \setlength{\parindent}{0pt}
    \setlength{\parskip}{6pt plus 2pt minus 1pt}}
}{% if KOMA class
  \KOMAoptions{parskip=half}}
\makeatother
\usepackage{xcolor}
\IfFileExists{xurl.sty}{\usepackage{xurl}}{} % add URL line breaks if available
\IfFileExists{bookmark.sty}{\usepackage{bookmark}}{\usepackage{hyperref}}
\hypersetup{
  pdftitle={No exit: the conservation reliance of species with too little habitat},
  hidelinks,
  pdfcreator={LaTeX via pandoc}}
\urlstyle{same} % disable monospaced font for URLs
\usepackage[margin=2.5cm]{geometry}
\usepackage{longtable,booktabs,array}
\usepackage{calc} % for calculating minipage widths
% Correct order of tables after \paragraph or \subparagraph
\usepackage{etoolbox}
\makeatletter
\patchcmd\longtable{\par}{\if@noskipsec\mbox{}\fi\par}{}{}
\makeatother
% Allow footnotes in longtable head/foot
\IfFileExists{footnotehyper.sty}{\usepackage{footnotehyper}}{\usepackage{footnote}}
\makesavenoteenv{longtable}
\usepackage{graphicx}
\makeatletter
\def\maxwidth{\ifdim\Gin@nat@width>\linewidth\linewidth\else\Gin@nat@width\fi}
\def\maxheight{\ifdim\Gin@nat@height>\textheight\textheight\else\Gin@nat@height\fi}
\makeatother
% Scale images if necessary, so that they will not overflow the page
% margins by default, and it is still possible to overwrite the defaults
% using explicit options in \includegraphics[width, height, ...]{}
\setkeys{Gin}{width=\maxwidth,height=\maxheight,keepaspectratio}
% Set default figure placement to htbp
\makeatletter
\def\fps@figure{htbp}
\makeatother
\setlength{\emergencystretch}{3em} % prevent overfull lines
\providecommand{\tightlist}{%
  \setlength{\itemsep}{0pt}\setlength{\parskip}{0pt}}
\setcounter{secnumdepth}{5}
\usepackage{float}
\usepackage{subcaption}
\usepackage{fancyhdr}
\pagestyle{fancy}
\fancyhf{}
\chead{Supplementary Information}
\fancyfoot[CO,CE]{\thepage}
\usepackage{setspace}
\doublespacing

 % For manual landscape pages
\usepackage{lscape}
\newcommand{\blandscape}{\begin{landscape}}
\newcommand{\elandscape}{\end{landscape}}

% For manual syntax highlighting
\usepackage{fontspec}
\setmainfont{Calibri} % Calibri is main font
\newcommand*{\myfont}{\fontfamily{lmtt}\selectfont} % Latin Modern typewriter for syntax
\DeclareTextFontCommand{\textmyfont}{\myfont}

%  For tables
\usepackage{multirow}
\usepackage{gensymb}

% To change labelling of supplementary figures and tables
\newcommand{\beginsupplement}{\setcounter{table}{0}  \renewcommand{\tablename}{Supplementary Table} \setcounter{figure}{0} \renewcommand{\figurename}{Supplementary Figure}}
\ifLuaTeX
  \usepackage{selnolig}  % disable illegal ligatures
\fi

\begin{document}
\maketitle

\begin{center}\rule{0.5\linewidth}{0.5pt}\end{center}

Sophia Richter\textsuperscript{1}‡, Rebecca A. Senior\textsuperscript{2,3}*‡, and David S. Wilcove\textsuperscript{1,2}

\textsuperscript{1}Department of Ecology and Evolutionary Biology, Princeton University, Princeton, NJ, USA.

\textsuperscript{2}Princeton School of Public and International Affairs, Princeton University, Princeton, NJ, USA.

\textsuperscript{3}Conservation Ecology Group, Department of Biosciences, Durham University, Durham DH1 3LE, UK

*\textbf{Corresponding author:} \href{mailto:rebecca.a.senior@gmail.com}{\nolinkurl{rebecca.a.senior@gmail.com}} (R.A. Senior)

‡These authors contributed equally to this work

\textbf{ORCID iDs:} orcid.org/0000-0002-8208-736X (R.A. Senior)

\pagebreak
\raggedright
\beginsupplement

\pagebreak



\begin{figure}

{\centering \includegraphics{figs/som/fig-s-1-1} 

}

\caption{The total number of threatened species within each country classified as `No Exit Species.' Results are summarised for each taxonomic group (rows), and shaded from low count (yellow) to high (red).}\label{fig:fig-s-1}
\end{figure}

\pagebreak



\begin{figure}

{\centering \includegraphics{figs/som/fig-s-2-1} 

}

\caption{The total number of threatened species within each country classified as `Extreme No Exit Species.' Results are summarised for each taxonomic group (rows), and shaded from low count (yellow) to high (red).}\label{fig:fig-s-2}
\end{figure}

\pagebreak



\begin{figure}

{\centering \includegraphics{figs/som/fig-s-3-1} 

}

\caption{The total number of threatened species within each zoogeographic realm classified as `No Exit Species.' Results are summarised for each taxonomic group (rows), and shaded from low count (yellow) to high (red).}\label{fig:fig-s-3}
\end{figure}

\pagebreak



\begin{figure}

{\centering \includegraphics{figs/som/fig-s-4-1} 

}

\caption{The total number of threatened species within each zoogeographic realm classified as `Extreme No Exit Species.' Results are summarised for each taxonomic group (rows), and shaded from low count (yellow) to high (red).}\label{fig:fig-s-4}
\end{figure}

\pagebreak



\begin{figure}

{\centering \includegraphics{figs/som/fig-s-5-1} 

}

\caption{The model-predicted relationship between species' proximity to threshold and absolute Area of Habitat (AOH), averaged across taxonomic classes and zoogeographic realms. Proximity to threshold is scaled from 0 to 1, with 0 representing the minimum difference between species' AOH and the downlisting threshold (i.e.~no difference), and 1 representing the maximum difference between species' AOH and the downlisting threshold (the value of the maximum difference is dependent on which downlisting threshold AOH is being compared to). Grey shading represents 95\% Confidence Intervals.}\label{fig:fig-s-5}
\end{figure}

\end{document}
